\section{\label{build}Building from sources}

This section describes how to build the {\CDI} library from the sources on a UNIX system.
{\CDI} is using the GNU configure and build system to compile the source code.
The only requirement is a working ANSI C99 compiler.

First go to the \href{https://code.mpimet.mpg.de/projects/cdi/files}{\texttt{download}} page
(\texttt{https://code.mpimet.mpg.de/projects/cdi/files}) to get the latest distribution,
if you do not already have it.

To take full advantage of {\CDI}'s features the following additional libraries should be installed:

\begin{itemize}
\item Unidata \href{http://www.unidata.ucar.edu/packages/netcdf}{NetCDF} library
      (\texttt{http://www.unidata.ucar.edu/packages/netcdf})
      version 3 or higher.
      This is needed to read/write NetCDF files with {\CDI}. 
\item ECMWF \href{https://software.ecmwf.int/wiki/display/ECC/ecCodes+Home}{ecCodes} library
      (\texttt{https://software.ecmwf.int/wiki/display/ECC/ecCodes+Home})
      version 2.3.0 or higher.
      This library is needed to encode/decode GRIB2 records with {\CDI}. 
\end{itemize}


\subsection{Compilation}

Compilation is now done by performing the following steps:

\begin{enumerate}
\item Unpack the archive, if you haven't already done that:
   
\begin{verbatim}
    gunzip cdi-$VERSION.tar.gz    # uncompress the archive
    tar xf cdi-$VERSION.tar       # unpack it
    cd cdi-$VERSION
\end{verbatim}

\item Run the configure script:
 
\begin{verbatim}
    ./configure
\end{verbatim}

Or optionally with NetCDF support:
 
\begin{verbatim}
    ./configure --with-netcdf=<NetCDF root directory>
\end{verbatim}

For an overview of other configuration options use

\begin{verbatim}
    ./configure --help
\end{verbatim}

\item Compile the program by running make:

\begin{verbatim}
    make
\end{verbatim}

\end{enumerate}

The software should compile without problems and the {\CDI} library (\texttt{libcdi.a}) 
should be available in the \texttt{src} directory of the distribution.


\subsection{Installation}

After the compilation of the source code do a \texttt{make install},
possibly as root if the destination permissions require that.

\begin{verbatim}
    make install
\end{verbatim} 

The library is installed into the directory \texttt{$<$prefix$>$/lib}.
The C and Fortran include files are installed into the directory \texttt{$<$prefix$>$/include}.
\texttt{$<$prefix$>$} defaults to \texttt{/usr/local} but can be changed with 
the \texttt{--prefix} option of the configure script. 

%Alternatively, you can also copy the library from the {\tt src} directory
%manually to some {\tt lib} directory in your search path.

%%% Local Variables: 
%%% mode: latex
%%% TeX-master: "usage"
%%% End: 
